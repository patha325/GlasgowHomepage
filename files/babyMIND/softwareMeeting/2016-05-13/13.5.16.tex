\documentclass{article}
\usepackage[english]{babel}
\usepackage{minutes}


\minutesstyle{
columns = {1},
header =  {list}, %or {table},
vote = {list}, %or {table},
contents = {true}, %or {false}
}

\title{Collection of minutes}
\author{\LaTeXe}

\begin{document}

\begin{Minutes}{babyMIND software meeting}
%%\subtitle{}
%%\moderation{}
\minutetaker{Sven-Patrik Hallsj\"o}

\participant{Saba Parsa, Ryan Bayes, Etam Messomo}
%\missing[with excuse]{no excuse}
%%\missingExcused{}
%%\missingNoExcuse{}
%\guest{}
\minutesdate{May 13, 2016}
%%\starttime{}
%%\endtime{}
\location{Skype}
%%\cc{}
\maketitle

\topic{Merging software}
Ryan Bayes asks, what conditions do we need to migrate to GitLab.
Sven-Patrik Hallsj\"o adds that we want to finalize some things.
Migrate it until Thursday (night) next week 19/5-16.

An open question, when will INR they upload code?

\topic{DAQ software}
Ryan Bayes adds that we want to integrate the DAQ with SaRoMaN.
What are we expecting for the beam tests?
From Wednesday meeting, want event display from the testbeams.
Same beam as the one MIND will be on. Would be handy, software side, to be able to provide a characteristic of the beam, depends on how many planes we have. PID using MICE EMR algorithms, will help for/with the MIND reconstruction. 

Etam Messomo, framework for the DAQ exists, can look at it. Firmware only gives hit amplitude as of now. 

Data protocoll has been presented previously.
16/3 Indeco.
\url{https://indico.cern.ch/event/493123/contributions/1171582/attachments/1244256/1831611/BabyMind_YF_Cern_16-03-2016.pdf}
and there are  scripts.

\topic{Current progress}
Saba Parsa, not working with simulations. Working at CERN.
Ryan Bayes, MICE took priortity 
Sven-Patrik Hallsj\"o, fixed fitted nodes, reconstructed momentum is lower than expected.

\topic{Next meeting}
Friday the 20/5-16 at 10 am BST via Skype.

\end{Minutes}

\end{document}