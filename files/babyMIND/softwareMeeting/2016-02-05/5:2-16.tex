\documentclass{article}
\usepackage[english]{babel}
\usepackage{minutes}


\minutesstyle{
columns = {1},
header =  {list}, %or {table},
vote = {list}, %or {table},
contents = {true}, %or {false}
}

\title{Collection of minutes}
\author{\LaTeXe}

\begin{document}

\begin{Minutes}{babyMIND software meeting}
%%\subtitle{}
%%\moderation{}
\minutetaker{Sven-Patrik Hallsj\"o}
\participant{Ryan Bayes, Etam Messomo, Saba Parsa.}
%\missing[with excuse]{no excuse}
%%\missingExcused{}
%%\missingNoExcuse{}
%\guest{}
\minutesdate{February 5, 2016}
%%\starttime{}
%%\endtime{}
\location{Skype}
%%\cc{}
\maketitle

\topic{Choosing a suitable meeting time for future meetings}%<-- insert title of topic
The meeting agreed on a meeting time for future meetings: Fridays 10-11 GMT.

\topic{Current limitations in the software}
Patrik discussed where the software package is currently and what limitations exist. These are formulated in a short introduction to the code and a ToDo list.

\topic{How to handle multiple developers on the code}
As of now two lists will be distributed, one with a short introduction to the code and one with a ToDo list of what is needed to be done with the code until a first release. These lists will be distributed to the developers. At a later stage a list of what simulations are needed will be distributed. 

It was agreed to setup a framwork where tasks can be easily handed out so that there is an overview of the code and its status.

\task{Sven-Patrik Hallsj\"o}{Create this frame work at Glasgow University.}

\topic{Other topics}

INR will present software status next main meeting. Will have to open a channel for discussion with them.

\topic{Next meeting}
Friday the 12/2-16 at 10 am GMT.

\end{Minutes}

\end{document}