\documentclass{article}
\usepackage[english]{babel}
\usepackage{minutes}


\minutesstyle{
columns = {1},
header =  {list}, %or {table},
vote = {list}, %or {table},
contents = {true}, %or {false}
}

\title{Collection of minutes}
\author{\LaTeXe}

\begin{document}

\begin{Minutes}{babyMIND software meeting}
%%\subtitle{}
%%\moderation{}
\minutetaker{Sven-Patrik Hallsj\"o}

\participant{Saba Parsa, Ryan Bayes }
%\missing[with excuse]{no excuse}
%%\missingExcused{}
%%\missingNoExcuse{}
%\guest{}
\minutesdate{April 22, 2016}
%%\starttime{}
%%\endtime{}
\location{Skype}
%%\cc{}
\maketitle

\topic{Geant visualizer}
Ryan Bayes gave a tutorial on how to use it.
vis.mac file (is on git) needs to be in working directory to be used.
File is well commented, everything in vis.mac can also be run in the command line.
Use beamOn x, where x is the number of events you wish to see.
after which use /vis/reviewKeptEvents to add them to the visualizer again.
with /vis/scene/add/trajectories vill add plot first event,
after with plot cont to plot out the next event.
Ryan Bayes will put his vis.mac and related code on github.
vis.mac is under /sciNDG4

\topic{Merging software}
Next meeting since the discussion requires Alexander

Ryan Bayes adds:
Would be nice to incorporate digitization, DAQ (not sure for the test beam)
Need to tell showers from tracks,
need to handle multiple hits per plane before June beam test.

\topic{Progress of framework for multiple developers}
\url{http://lspace.ppe.gla.ac.uk/}
Patrik discusses the gitlab that exists, how to create accounts and what is there.
Will be used in the future.

\topic{Next meeting}
Friday the 13/5-16 at 10 am BST via Skype.

\end{Minutes}

\end{document}