\documentclass{article}
\usepackage[english]{babel}
\usepackage{minutes}


\minutesstyle{
columns = {1},
header =  {list}, %or {table},
vote = {list}, %or {table},
contents = {true}, %or {false}
}

\title{Collection of minutes}
\author{\LaTeXe}

\begin{document}

\begin{Minutes}{babyMIND software meeting}
%%\subtitle{}
%%\moderation{}
\minutetaker{Sven-Patrik Hallsj\"o}
\participant{Ryan Bayes, Etam Messomo, Saba Parsa.}
%\missing[with excuse]{no excuse}
%%\missingExcused{}
%%\missingNoExcuse{}
%\guest{}
\minutesdate{February 12, 2016}
%%\starttime{}
%%\endtime{}
\location{Skype}
%%\cc{}
\maketitle

\topic{Discuss the release, limitations and status}%<-- insert title of topic
Sven-Patrik Hallsj\"o informed that the Glasgow University software will make a release today. It still have the reconstruction issue between $\mu^+$ and $\mu^-$, also a limit that tracks that pass through the detector can not be reconstructed. 

\topic{Progress of framework for multiple developers}
Sven-Patrik Hallsj\"o informed that Glasgow University is working on creating a framework preliminarily using Jenkins and redmine to simplify development using multiple developers. 

At the moment the code can be downloaded publicly from github. A discussion arose that the code should always be available for the whole collaboration.

\topic{Discuss this weeks update meeting}
The main discussion point was the presentation given by Alexander Izmaylov at the Baby MIND update meeting on Wednesday the 10th of February 2016.
Ryan Bayes and Sven-Patrik Hallsj\"o  discussed that from slide 3 of the given presentation the following things would be of interest.
\begin{itemize}
\item In the Glasgow software we are currently using a common GDML description for the geometry in the simulation and reconstruction. We think that this is a good way to proceed. The slides that were presented makes it look like the geometry in the INR software is independent of the geometry in the reconstruction.
\item It would be good to take some parts from the code and incorporate it in to our code. This merge needs to be discussed with INR.
\item Data handling is not incorporated in the Glasgow software. Do we want a MIDAS based system for the test beam? Needs to be discussed with the collaboration.
\item A merge would be good as to not duplicate effort. We have implemented a Python wrapper and the use of a GDML geometry which simplifies making geometry changes and simplifies usage. Suggestion that Glasgow can handle the merge.
\end{itemize}

Etam Messomo made the suggestion that Glasgow should handle the setup of all the software for babyMIND since they already have a public Github.  
\task{Alexander Izmaylov}{How can people not in T2K access your software?}

\topic{Next meeting}
Friday the 19/2-16 at 10 am GMT via Skype.

\end{Minutes}

\end{document}